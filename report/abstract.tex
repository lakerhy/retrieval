\begin{abstract}
\thispagestyle{empty} \noindent{}
A free things has to be done in this part especially in the field of 
% The debate about pros and cons of
% open source software and proprietary software has increased
% remarkable the last few years. Today we see an increasing number
% of software solutions based on open source software and it appears
% that more are to come. This come true due to the fact that big
% players in the software market have adjusted their business models
% to meet the requirements of open source software. This report
% looks at the theoretical aspects that forms different business
% models used in the software market.

% Empirical cases from the software industry involves studies on
% Microsoft, Netscape, Apache, Red Hat and other interested parties.
% It is shown how software companies can make money on producing
% proprietary software, changing from proprietary software to open
% source software or producing open source software.

% Finally a discussion of the different business models and
% approaches with respect to the theory and the analyzed cases are
% presented. We find that there are many ways to expand the profit
% when producing open source software. We also find that although
% open source software is becoming more and more mature, there is a
% demand for major players to vouch for complete software systems.
% Another finding is that despite open source software is free to
% download, the total cost of ownership might equal the expenses of
% proprietary software, seen from a users point of view. From a
% developers point of view, there can be several advantages of
% switching to develop open source software, such as lower
% development cost, risk spreading etc. The disadvantages include
% organization and implementation problems, and -- in some cases --
% unmature software. We believe that business models based on open
% source software will overcome many of these disadvantages in the
% future, and that it will turn out to be a indispensable competitor
% to proprietary software.

\paragraph{Keywords:} MEMLS, Model, Thing
\end{abstract}
%%% Local Variables: 
%%% mode: latex
%%% TeX-master: "main"
%%% End: 
